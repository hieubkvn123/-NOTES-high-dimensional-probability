\subsection{Leibniz Differentiation Rule}
\label{sec:leib_diff_rule}
\textbf{Leibniz Differentiation/Integral Rule} refers to the conditions by which differentiation and integration can be exchanged. Essentially, let $\Omega$ be a measure space and $A\subset\R^n$ be an open subset. Then, if a function $f:A \times \Omega \to \R$ satisfies:
\begin{enumerate}
	\item For all $x\in A$, $\omega\mapsto f(x, \omega)$ is a measurable function in $\omega$ ($f$ is a \underline{Caratheodory Function}).	
	\item For almost all $\omega\in\Omega$, $\frac{\partial f(x, \omega)}{\partial x}$ exists for all $x\in A$.
	\item For almost all $\omega\in\Omega$, there is an integrable function $h:\Omega\to\R$ such that $|f(x, \omega)|\le h(x)$ for all $x\in A$.
\end{enumerate} 

\begin{definition}[Caratheodory Function]
	Let $(\Omega, \mathcal{F}, \mu)$ be a measure space and $A\subset\R^n$. We say $f: A \times \Omega \to \R$ is a Caratheodory Function if it satisfies:
	\begin{enumerate}
		\item For all $x\in A$: $\omega\mapsto f(x, \omega)$ is $\mathcal{F}$-measurable.	
		\item For all $\omega\in\Omega$: $x\mapsto f(x, \omega)$ is continuous.
	\end{enumerate}

	\noindent We also say that $f$ is a Caratheodory function if it is measurable in $\omega$ and continuous in $x$.
\end{definition} 
