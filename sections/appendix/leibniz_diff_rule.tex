\subsection{Leibniz Differentiation Rule}
\label{sec:leib_diff_rule}

\textbf{Leibniz Differentiation/Integral Rule} refers to the conditions by which differentiation and integration can be exchanged. Essentially, let $\Omega$ be a measure space and $A\subset\R^n$ be an open subset. Then, if a function $f:A \times \Omega \to \R$ satisfies:
\begin{enumerate}
	\item For all $x\in A$, $\omega\mapsto f(x, \omega)$ is a measurable function in $\omega$ ($f$ is a \underline{Caratheodory Function}).	
	\item For almost all $\omega\in\Omega$, $\frac{\partial f(x, \omega)}{\partial x}$ exists for all $x\in A$.
	\item For almost all $\omega\in\Omega$, there is an integrable function $h:\Omega\to\R$ such that $|f(x, \omega)|\le h(\omega)$ for all $x\in A$.
\end{enumerate} 

\noindent Then, we have:
\begin{enumerate}
	\item $g(x) = \int_\Omega f(x, \omega)d\mu(\omega)$ is continuous.
	\item $D_i g(x) = \int_\Omega D_i f(x, \omega)d\mu(\omega)$.
\end{enumerate} 


\begin{definition}[Caratheodory Function]
	Let $(\Omega, \mathcal{F}, \mu)$ be a measure space and $A\subset\R^n$. We say $f: A \times \Omega \to \R$ is a Caratheodory Function if it satisfies:
	\begin{enumerate}
		\item For all $x\in A$: $\omega\mapsto f(x, \omega)$ is $\mathcal{F}$-measurable.	
		\item For all $\omega\in\Omega$: $x\mapsto f(x, \omega)$ is continuous.
	\end{enumerate}

	\noindent We also say that $f$ is a Caratheodory function if it is measurable in $\omega$ and continuous in $x$.
\end{definition} 

\begin{proposition}{Leibniz Differentiation Rule}{leibniz_diff_rule}
	Let $(\Omega, \mathcal{F}, \mu)$ be a measure space and $A\subset\R^n$ be open. Let the function $f: A\times\Omega\to\R$ satisfy the following:
	\begin{enumerate}
		\item $f$ is a Caratheodory function.
		\item There exists $h: \Omega\to\R$ such that $|f(x, \omega)|\le h(\omega)$ for almost all $\omega\in\Omega$ for all $x\in A$.
	\end{enumerate}

	\noindent Then, if we define $g(x) = \int_\Omega f(x, \omega)d\mu(\omega)$, we have:
	\begin{enumerate}[label=(\roman*)]
		\item $g$ is continuous.
		\item $D_i g(x) = \int_\Omega D_i f(x, \omega)d\mu(\omega)$.
	\end{enumerate} 
\end{proposition} 

\begin{proof*}[Proposition \ref{prop:leibniz_diff_rule}]
	We prove the above points one by one:
	\begin{enumerate}[label=(\roman*)]
		\item Suppose that $\{x_n\}_{n=1}^\infty\subset A$ is a sequence such that $x_n\to x\in A$. We have to prove that $g(x_n)\to g(x)$. Since $f$ is a Caratheodory function, we have $x\mapsto f(x, \omega)$ is continuous for almost all $\omega\in\Omega$. Therefore, $f(x_n,\omega)\to f(x, \omega)$ for almost all $\omega\in\Omega$. Then, by the Dominated Convergence Theorem\footnote{DCT is applied for the sequence of functions $\{k_n\}_{n=1}^\infty$ where $k_n:\Omega\to\R$ and $k_n(\omega)=f(x_n, \omega)$.}:
		\begin{align*}
			\int_\Omega f(x_n, \omega) d\mu(\omega) \to \int_\Omega f(x, \omega)d\mu(\omega).	
		\end{align*} 

		\noindent Hence, we have $g(x_n) \to g(x)$ as desired.

		\item Let $e_i$ denotes the $i^{th}$ standard basis in $\R^n$. We have:
		\begin{align*}
			D_i f(x, \omega) = \lim_{\xi\to 0} \frac{f(x+\xi e_i, \omega) - f(x, \omega)}{\xi}.
		\end{align*} 
	\end{enumerate} 	
\end{proof*} 
