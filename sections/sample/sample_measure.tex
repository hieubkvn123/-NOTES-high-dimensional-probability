\section{Overview of measure theory}
\subsection{Algebra and $\sigma$-algebra}
\begin{definition}[Algebra]
    Let $X$ be a set and $\mathcal{A}$ be a collection of subsets of $X$. Then, we say that $\mathcal{A}$ is an algebra if it satisfies:
    \begin{itemize}
        \item \textbf{Closure under complement} : If $E \in \mathcal{A} \implies E^c \in \mathcal{A}$.
        \item \textbf{Closure under finite union} : For all finite collection $\{E_n\}_{n=1}^N\subset \mathcal{A} \implies \bigcup_{n=1}^N E_n \in \mathcal{A}$.
    \end{itemize}
\end{definition}

\begin{definition}[$\sigma$-algebra]
    Let $X$ be a set and $\mathcal{A}$ be a collection of subsets of $X$. Then, we say that $\mathcal{A}$ is a $\sigma$-algebra if it is:
    \begin{itemize}
        \item \textbf{Closure under complement} : If $E \in \mathcal{A} \implies E^c \in \mathcal{A}$.
        \item \textbf{Closure under countable union} : For all countable collection $\{E_n\}_{n=1}^\infty\subset \mathcal{A} \implies \bigcup_{n=1}^\infty E_n \in \mathcal{A}$.
    \end{itemize}
\end{definition}

\begin{definition}[Borel-$\sigma$-algebra]
    Let $\Sigma$ be the set of all the $\sigma$-algebras generated by open intervals in $\mathbb{R}$. Then, the Borel-$\sigma$-algebra is the smallest $\sigma$-algebra containing the open intervals:
    \begin{align*}
        \mathcal{B} = \bigcap_{\mathcal{A}\in\Sigma} \mathcal{A}
    \end{align*}
\end{definition}

\begin{proposition}{Disjoint union in algebra}{disjoint_union_in_algebra}
    Let $\mathcal{A}$ be an algebra and let $\{E_n\}_{n=1}^\infty$ be a countable collection of subsets in $\mathcal{A}$. Then, there exists a countable disjoint subsets $\{F_n\}_{n=1}^\infty$ such that:
    \begin{align*}
        \bigcup_{n=1}^\infty E_n = \bigcup_{n=1}^\infty F_n
    \end{align*}
\end{proposition}

\begin{proof*}[Proposition \ref{prop:disjoint_union_in_algebra}]
    Let $G_m = \bigcup_{n=1}^m E_n$, we have $G_1 \subset G_2 \subset G_3 \subset \dots \subset G_N$. It is easy to see that $\bigcup_{n=1}^N G_n = \bigcup_{n=1}^N E_n$. Now, define the collection $\{F_n\}_{n=1}^\infty$ as followed:
    \begin{align*}
        F_n = 
        \begin{cases}
            G_1 & \text{When } n = 1
            \\ \\
            G_{n} \setminus G_{n-1} & \text{When } n \ge 2
        \end{cases}
    \end{align*}

    \noindent Hence, we have $\bigcup_{n=1}^N F_n = \bigcup_{n=1}^N G_n \implies \bigcup_{n=1}^NF_n = \bigcup_{n=1}^N E_n$.
\end{proof*}

\subsection{Lebesgue measure}
\noindent \textbf{Overview} : The definition of Lebesgue measure stems from the need to construct a more general notion of integral (the Lebesgue integral) because the simple notion of Riemann integral is incomplete. For example, $L^1_R([0,1])$ (space of absolutely Riemann-integrable functions) is not a Banach space.

\noindent\newline The construction of the Lebesgue integral over $\mathbb{R}$ requires a notion of "measure" on subsets of $\mathbb{R}$, which, ideally satisfies the following conditions:
\begin{itemize}
    \item $\mu:\mathcal{P}(\mathbb{R}) \to [0, \infty)$ where $\mathcal{P}(\mathbb{R})$ denotes the power set of $\mathbb{R}$.
    \item $\mu$ \textbf{extends the measure of interval length} $l$. Meaning, if $I\subset \mathbb{R}$ is an interval, $\mu(I)=l(I)$.
    \item \textbf{Countable additivity} : Let $\{E_n\}_{n=1}^\infty\subset X$ be a collection of disjoint subsets of $\mathbb{R}$, then $\mu(\bigcup_{n=1}^\infty E_n) = \sum_{n=1}^\infty \mu(E_n)$.
    \item \textbf{Translation invariance} : For $E \subset \mathbb{R}, x\in \mathbb{R}$, we have $\mu(E+x)=\mu(E)$.
\end{itemize}

\noindent However, it is widely known that the construction of such measure is not possible because of the existence of non-measurable sets (Vitali sets \cite{wiki:vitaliset}).

\begin{definition}[Lebesgue outer measure]
    Let $E\subset \mathbb{R}$. The Lebesgue outer measure (or simply "outer measure") is a mapping from the power set of $\mathbb{R}$ to $[0, \infty)$ such that:
    \begin{align*}
        \mu^*(E) = \inf\Bigg\{ \sum_{n=1}^\infty l(I_n) : I_n \text{ are open intervals}; E \subseteq \bigcup_{n=1}^\infty I_n \Bigg\}
    \end{align*}

    \noindent Where $l$ denotes interval length. Without proving, we will just acknowledge the fact that the Lebesgue outer measure satisfies the second and the fourth conditions. However, \textbf{the outer measure is countably sub-additive rather than countably additive}. To account for this, we look at the definition of the Caratheodory criterion below.
\end{definition}

\begin{definition}[Caratheodory criterion - Lebesgue measurable sets]
    Let $E \subseteq \mathbb{R}$. The set $E$ is called \textbf{Lebgesgue measurable} if for all $A\subseteq \mathbb{R}$, we have:
    \begin{align*}
        \mu^*(A) = \mu^*(A \cap E) + \mu^*(A\cap E^c)
    \end{align*}
    \noindent The above condition is called the \textbf{Caratheodory criterion}. We denote the set of Lebesgue measurable subsets as $\mathcal{M}$:
    \begin{align*}
        \mathcal{M} = \Bigg\{ E \subseteq \mathbb{R} : \forall A \subseteq \mathbb{R}, \mu^*(A) = \mu^*(A \cap E) + \mu^*(A\cap E^c) \Bigg\}
    \end{align*}
\end{definition}

\noindent \newline \textbf{Remark} : Note that by countable sub-additivity, we will always have $\mu^*(A) \ge \mu^*(A\cap E)$

\begin{definition}[Lebesgue measure]
    The Lebesgue measure (denoted $\mu$) is simply the Lebesgue outer measure $\mu^*$ restricted to the set of Lebesgue measurable sets $\mathcal{M}$:
    \begin{align*}
        \mu : \mathcal{M} \to [0, \infty); \ \ \mu := \mu^*\Big|_{\mathcal{M}}
    \end{align*}
\end{definition}

\begin{proposition}{Measure of intersection with measurable collection}{intersection_with_measurable_collection}
    Let $A\subseteq \mathbb{R}$ and let $\{E_n\}_{n=1}^N$ be a finite disjoint collection of Lebesgue measurable sets. Then, we have:
    \begin{align*}
        \mu^*\Bigg( A \cap \Bigg[ \bigcup_{n=1}^N E_n \Bigg] \Bigg) = \sum_{n=1}^N \mu^*(A\cap E_n)
    \end{align*}
\end{proposition}

\begin{proof*}[Proposition \ref{prop:intersection_with_measurable_collection}]
    We will prove this by induction. For $N=1$, both sides are identical. For the inductive step, suppose that the above proposition is true for $N=m$. We have to prove that it is true for $N=m+1$.

    \noindent \newline Since $E_{m+1}$ is measurable, using the Caratheodory criterion, we have:
    \begin{align*}
        \mu^*\Bigg( A \cap \Bigg[ \bigcup_{n=1}^{m+1} E_n \Bigg] \Bigg) 
            &= \mu^*\Bigg( A \cap \Bigg[ \bigcup_{n=1}^{m+1} E_n \Bigg] \cap E_{m+1} \Bigg)
            + \mu^*\Bigg( A \cap \Bigg[ \bigcup_{n=1}^{m+1} E_n \Bigg] \cap E_{m+1}^c \Bigg) \\
            &= \mu^*(A\cap E_{m+1}) + \mu^*\Bigg( A \cap \Bigg[ \bigcup_{n=1}^{m+1} E_n \Bigg] \cap E_{m+1}^c \Bigg)
    \end{align*}

    \noindent \newline Since $E_{n}$ is disjoint for all $1 \le n \le m+1$. We have:
    \begin{align*}
        \bigcup_{n=1}^{m} E_n \subset E_{m+1}^c \implies \Bigg[ \bigcup_{n=1}^{m+1} E_n \Bigg] \cap E_{m+1}^c = \bigcup_{n=1}^m E_n
    \end{align*}

    \noindent \newline Finally, we have
    \begin{align*}
        \mu^*\Bigg( A \cap \Bigg[ \bigcup_{n=1}^{m+1} E_n \Bigg] \Bigg) 
            &= \mu^*(A\cap E_{m+1}) + \mu^*\Bigg( A \cap \Bigg[ \bigcup_{n=1}^{m} E_n \Bigg] \Bigg) \\
            &= \mu^*(A\cap E_{m+1}) + \sum_{n=1}^m \mu^*(A\cap E_n) \\
            &= \sum_{n=1}^{m+1} \mu^*(A\cap E_n)
    \end{align*}
\end{proof*}

\begin{proposition}{$\mathcal{M}$ is $\sigma$-algebra}{m_is_sigma_algebra}
    The set of Lebesgue measurable subsets $\mathcal{M}$ is a $\sigma$-algebra.
\end{proposition}
\begin{proof*}[Proposition \ref{prop:m_is_sigma_algebra}]
    We first prove that $\mathcal{M}$ is an algebra. Then, for all countable collection of Lebesgue measurable sets $\{E_n\}_{n=1}^\infty$ such that $E = \bigcup_{n=1}^\infty E_n$, $\mu^*(A)\ge\mu^*(A\cap E) + \mu^(A\cap E^c)$.
    
    \begin{subproof}{\newline Claim 1 : $\mathcal{M}$ is an algebra}
        We have to prove that $\mathcal{M}$ is both closed under complement and finite union.
        \begin{itemize}
            \item \textbf{Closure under complement} : Trivial due to the symmetry of the Caratheodory criterion.
            \item \textbf{Closure under finite union} : Let $E_1, E_2$ be two Lebesgue measurable sets. We have:
        
            \begin{align*}
                \mu^*(A\cap(E_1\cup E_2)) &= \mu^*((A \cap E_1) \cup (A\cap E_2)) \\
                &=   \mu^*((A \cap E_1) \cup (A\cap E_2 \cap E_1^c)) \\
                &\le \mu^*(A\cap E_1) + \mu^*(A\cap E_2 \cap E_1^c) \ \ \ \text{(Countable sub-additivity)} \\
                &=   \mu^*(A) - \mu^*(A\cap E_1^c) + \mu^*(A\cap E_2 \cap E_1^c) \\
                &=   \mu^*(A) - \Big[ \mu^*(A\cap E_1^c) - \mu^*([A\cap E_1^c] \cap E_2) \Big] \\
                &=   \mu^*(A) - \mu^*(A\cap E_1^c \cap E_2^c) = \mu^*(A) - \mu^*\Big( A \cap [E_1\cup E_2]^c \Big) \\
                \implies \mu^*(A) &\ge \mu^*\Big(A\cap(E_1\cup E_2)\Big) + \mu^*\Big( A \cap [E_1\cup E_2]^c \Big) \\
                \implies & E_1 \cap E_2 \in \mathcal{M}
            \end{align*}
        \end{itemize}
    \end{subproof}

    \begin{subproof}{\newline Claim 2 : $\mathcal{M}$ is a $\sigma$-algebra}
        Given $\{E_n\}_{n=1}^\infty$ be a countable collection of Lebesgue measurable sets and let $E=\bigcup_{n=1}^\infty E_n$. By proposition \ref{prop:disjoint_union_in_algebra}, there exists another countable \textbf{disjoint} collection of Lebesgue measurable sets $\{F_n\}_{n=1}^\infty$ such that $\bigcup_{n=1}^\infty F_n = \bigcup_{n=1}^\infty E_n=E$. 
        
        \noindent \newline For any integer $N\ge1$, we have $\bigcup_{n=1}^N F_n$ is Lebesgue measurable because $\mathcal{M}$ is an algebra. Hence, we have:
    
        \begin{align*}
            \mu^*(A) &= \mu^*\Bigg( A \cap \Bigg[ \bigcup_{n=1}^N F_n \Bigg] \Bigg) + \mu^*\Bigg( A \cap \Bigg[ \bigcup_{n=1}^N F_n \Bigg]^c \Bigg) \\
            &\ge \mu^*\Bigg( A \cap \Bigg[ \bigcup_{n=1}^N F_n \Bigg] \Bigg) + \mu^*\Bigg( A \cap \Bigg[ \bigcup_{n=1}^\infty F_n \Bigg]^c \Bigg) = \mu^*\Bigg( A \cap \Bigg[ \bigcup_{n=1}^N F_n \Bigg] \Bigg) + \mu^*(A\cap E^c)
        \end{align*}

        \noindent\newline By proposition \ref{prop:intersection_with_measurable_collection}, we have:
        \begin{align*}
            \mu^*(A) &\ge \sum_{n=1}^N \mu^*(A\cap F_n) + \mu^*(A\cap E^c)
        \end{align*}

        \noindent\newline Taking $N\to\infty$, we have:
        \begin{align*}
            \mu^*(A) &\ge \sum_{n=1}^\infty \mu^*(A\cap F_n) + \mu^*(A\cap E^c) \\
            &= \mu^*\Bigg( A \cap \Bigg[ \bigcup_{n=1}^\infty F_n \Bigg]\Bigg) + \mu^*(A\cap E^c) = \mu^*(A\cap E) + \mu^*(A\cap E^c)
        \end{align*}

        \noindent\newline Hence, $\mathcal{M}$ is closed under countable union and is a $\sigma$-algebra.
    \end{subproof}
\end{proof*}



\subsection{Measurable spaces}
\begin{definition}[Measurable space]
    Let $E$ be a set and $\mathcal{E}$ be a $\sigma$-algebra over $E$. Then, the pair $(E, \mathcal{E})$ is called a \textbf{measurable space}. The elements in $\mathcal{E}$ are called \textbf{measurable sets}. When $E$ is a topological space and $\mathcal{E}$ is the Borel-$\sigma$-algebra on $E$, then the elements in $\mathcal{E}$ are also called \textbf{Borel sets}.
\end{definition}

\begin{definition}[Product of measurable spaces]
    Let $(E, \mathcal{E})$ and $(F, \mathcal{F})$ be measurable spaces. For $A\subset E, B\subset F$, we denote the product of $A, B$, denoted $A\times B$, as the set of all pairs $(x, y)$ such that $x \in A, y\in B$. The set $A\times B$ is then called a \textbf{measurable rectangle}. The measurable space $(E\times F, \mathcal{E} \otimes \mathcal{F})$ is called the product of measurable spaces $(E, \mathcal{E})$ and $(F, \mathcal{F})$ where $\mathcal{E} \otimes \mathcal{F}$ is a $\sigma$-algebra over $E\times F$:
    \begin{align*}
        \mathcal{E} \otimes \mathcal{F} = \Big\{ A \times B: A \in \mathcal{E}, B \in \mathcal{F} \Big\}
    \end{align*}
\end{definition}

\begin{definition}[Measure \& Measure space]
    Let $(E, \mathcal{E})$ be a measurable space. A measure is a mapping $\mu:\mathcal{E}\to[0, \infty]$ (Including infinity) such that:
    \begin{itemize}
        \item \textbf{Empty set has zero measure} : $\mu(\emptyset) = 0$.
        \item \textbf{Countable additivity} : For a collection of disjoint measurable sets $\{E_n\}_{n=1}^\infty$, we have
        \begin{align*}
            \mu\Bigg( \bigcup_{n=1}^\infty E_n \Bigg) = \sum_{n=1}^\infty \mu(E_n)
        \end{align*}
    \end{itemize}
\end{definition}

\noindent\newline \textbf{Remark} : Note that \textbf{monotonicity} and \textbf{translation invariance} are properties specific to Lebesgue measure only. A general measure need not to have those properties.  

\subsection{Measurable functions}
\begin{definition}[Function (Mapping)]
    A \textbf{mapping} or a \textbf{function} $f:E\to F$ from $E$ to $F$ is a rule that assigns every element $f(x) \in F$ to an element $x\in E$. Given a subset $B\subset F$, the pre-image of $B$ under a function $f:E\to F$ is given by:
    \begin{align*}
        f^{-1}(B) = \Big\{ x \in E : f(x) \in B \Big\}
    \end{align*}
\end{definition}

\begin{proposition}{Properties of functions}{props_of_functions}
    Let $f:E\to F$ be a function, the following properties hold:
    \begin{itemize}
        \item $f^{-1}(\emptyset)=\emptyset$.
        \item $f^{-1}(F)=E$.
        \item $f^{-1}(B\setminus C) = f^{-1}(B)\setminus f^{-1}(C)$.
        \item $f^{-1}\Bigg( \bigcup_{n=1}^\infty B_n \Bigg) = \bigcup_{n=1}^\infty f^{-1}(B_n)$.
        \item $f^{-1}\Bigg( \bigcap_{n=1}^\infty B_n \Bigg) = \bigcap_{n=1}^\infty f^{-1}(B_n)$.
    \end{itemize}
\end{proposition}

\begin{definition}[Measurable function]
    Let $(E, \mathcal{E})$ and $(F, \mathcal{F})$ be measurable spaces, $f:E\to F$ be a function. Then, $f$ is called \textbf{measurable relative to $\mathcal{E}$ and $\mathcal{F}$} (or $(\mathcal{E}, \mathcal{F})$-measurable) if:
    \begin{align*}
        \forall B \in \mathcal{F} : f^{-1}(B) \in \mathcal{E}
    \end{align*}
\end{definition}


