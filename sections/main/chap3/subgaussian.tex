\subsection{Sub-Gaussian Distributions}
A random variable $X$ is called ``sub-Gaussian" if its distribution has a strong tail decay similar to that of a Gaussian distribution. More precisely, the tail of $X$'s probability density/mass function is dominated by the tail of a Gaussian distribution, hence the name ``sub-Gaussian".

\noindent\newline Sub-Gaussian is a particularly interesting family of random variables because by modelling random events using sub-Gaussian distributions, we can bound the probability of very rare events happening at the tail. Furthermore, sub-Gaussianity implies the following properties:
\begin{enumerate}[label=(\roman*)]
	\item Bounded \textbf{tail probability}.
	\item Bounded \textbf{moment}.
	\item Bounded \textbf{moment generating function} (for $X$ and $X^2$).
	\item Bounded \textbf{sub-Gaussian norm\hyperref[sec:subgaussian_norm]{*}}.
\end{enumerate} 

\begin{definition}[Sub-Gaussian Random Variable]
	\label{sec:subgaussian_var}
	A random variable $X$ is called sub-Gaussian if there exists a constant $C>0$ such that:
	\begin{equation}
		\Pm(|X|\ge t) \le 2\exp(-t^2/C^2).	
	\end{equation} 
\end{definition} 

\begin{definition}[Sub-Gaussian Norm]
	\label{sec:subgaussian_norm}
	Let $X$ be a random variable. Then, we denote the sub-Gaussian norm of $X$ as:
	\begin{equation}
		\|X\|_{\psi_2} = \inf\bigCurl{
			t > 0: \E\bigSquare{\exp\bigRound{X^2/t^2}} \le 2
		}.
	\end{equation} 

	\noindent From the above definition and the definition of sub-Gaussian variables, we can deduce that $X$ is sub-Gaussian if and only if $\|X\|_{\psi_2} < \infty$. Furthermore, the intuition behind $\|.\|_{\psi_2}$ norm is that it measures the light-tailed-ness of a random variable. In other words, \textbf{the smaller the sub-Gaussian norm, the faster the tail decays}.
\end{definition} 

\begin{definition}[Sub-Gaussian Variance Proxy]
	Let $X$ be a sub-Gaussian random variable. Then, we say that $X$ is sub-Gaussian with variance proxy $\sigma^2$ if:
	\begin{equation}
		M_X(\lambda) \le \exp(\lambda^2\sigma^2/2), \quad \forall \lambda\in\R.	
	\end{equation} 

	\noindent We denote that $X\in\SG(\sigma^2)$.
\end{definition} 

\begin{proposition}{Properties of Sub-Gaussian Variables}{props_subgaussian}
	Let $X$ be a random variable with mean $\mu$. Then, the following properties are equivalent:

	\begin{enumerate}[label=(\roman*)]
		\item \textbf{Sub-Gaussianity (bounded tail probability)}: There exists $K_1>0$ such that
		\begin{equation}
			\Pm(|X| \ge t) \le 2\exp\biggRound{-\frac{t^2}{K_1^2}}.		
		\end{equation} 	

		\item \textbf{Bounded $p$-moment}: There exists $K_2>0$ such that
		\begin{equation}
			\|X\|_{L^p} = (\E |X|^p)^{1/p} \le K_2\sqrt{p}, \quad \forall p \ge 1.	
		\end{equation} 

		\item \textbf{Bounded MGF (of $X^2$)}: There exists $K_3>0$ such that
		\begin{equation}
			M_{X^2}(\lambda^2) \le \exp(K_3^2\lambda^2), \quad \forall \lambda\in\R \text{ and } |\lambda| \le \frac{1}{K_3}.	
		\end{equation} 

		\item \textbf{Bounded MGF (of $X$)}: There exists $K_4>0$ such that
		\begin{equation}
			M_{X-\mu}(\lambda) \le \exp(K_4^2\lambda^2), \quad \forall \lambda\in\R.
		\end{equation}

		\noindent In other words, $X-\mu$ has a variance proxy of $\sigma^2\le 2K_4^2$.

		\item \textbf{Bounded sub-Gaussian norm}: There exists $K_5>0$ such that
		\begin{equation}
			\E\bigSquare{\exp\bigRound{X^2/K_5^2}}\le 2.
		\end{equation} 

		\noindent In other words, $\|X\|_{\psi_2}\le K_5$.
	\end{enumerate} 

	\noindent The parameters $K_1, \dots, K_5$ differ from each other by at most an absolute constant. Meaning, there exists a constant $C$ independent of $K_1, \dots, K_5$ such that $K_i \le C K_j$ for any two $i,j \in \{1, \dots, 5\}$.
\end{proposition} 

\begin{proof*}[Proposition \ref{prop:props_subgaussian}]
	We prove that ${\bf (i)}\implies{\bf (ii)}\implies\dots\implies{\bf (v)}$ then prove that ${\bf (v)}\implies{\bf (i)}$.
	\begin{enumerate}
		\item ${\bf (i)}\implies{\bf (ii)}$: By the integral identity for $p^{th}$ moments, we have
		\begin{align*}
			\E|X|^p &= \int_0^\infty pt^{p-1}\Pm(|X|>t)dt \\
				&\le 2p\int_0^\infty t^{p-1}\exp\biggRound{-\frac{t^2}{K_1^2}}dt		
		\end{align*} 	
	\end{enumerate} 
\end{proof*} 
