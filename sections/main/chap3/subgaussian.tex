\subsection{Sub-Gaussian Distributions}
A random variable $X$ is called ``sub-Gaussian" if its distribution has a strong tail decay similar to that of a Gaussian distribution. More precisely, the tail of $X$'s probability density/mass function is dominated by the tail of a Gaussian distribution, hence the name ``sub-Gaussian".

\noindent\newline Sub-Gaussian is a particularly interesting family of random variables because by modelling random events using sub-Gaussian distributions, we can bound the probability of very rare events happening at the tail.

\begin{definition}[Sub-Gaussian Random Variable]
	A random variable $X$ is called sub-Gaussian if there exists a constant $C>0$ such that:
	\begin{equation}
		\Pm(|X|\ge t) \le 2\exp(-t^2/C^2).	
	\end{equation} 
\end{definition} 

\begin{definition}[Sub-Gaussian Variance Proxy]
	Let $X$ be a sub-Gaussian random variable. Then, we say that $X$ is sub-Gaussian with variance proxy $\sigma^2$ if:
	\begin{equation}
		M_X(\lambda) \le \exp(\lambda^2\sigma^2/2), \quad \forall \lambda\in\R.	
	\end{equation} 

	\noindent We denote that $X\in\SG(\sigma^2)$.
\end{definition} 

\begin{proposition}{Properties of Sub-Gaussian Variables}{props_subgaussian}
	
\end{proposition} 