\subsection{Basic inequalities}
First, we revisit the definition of a random variable as well as some basic inequalities that we
learned in introductory statistics.

\begin{definition}{Random variable}
    Let $(\Omega, \Sigma. \Pm)$ be a probability space. A random variable $X$ is defined as a mapping from the sample space $\Omega$ to $\R$:
    \begin{align}
        X:\Omega \to \R
    \end{align}

    \noindent $\Sigma$ is the $\sigma$-algebra containing the possible events (collection of subsets of $\Omega$) and $\Pm$ is a probability measure that assigns events with probabilities:
    \begin{align}
        \Pm: \Sigma \to [0,1]
    \end{align}
\end{definition}

\noindent For a given probability space $(\Omega, \Sigma, \Pm)$ and a random variable $X:\Omega\to\R$, we will use the following basic notations throughout this note:
\begin{itemize}
    \item $\|X\|_{L^p}$ - The $p^{th}$ root of the $p^{th}$ moment of the random variable $X$.
    \begin{align}
        \|X\|_{L^p} &= (\E|X|^p)^{1/p}, \ p \in (0,\infty) \\
        \|X\|_{L^\infty} &= \esssup |X|
    \end{align}

    \item $L^p(\Omega, \Sigma, \Pm)$ -  The space of random variables $X$ satisfying:
    \begin{align}
        L^p(\Omega, \Sigma, \Pm) = \bigCurl{
            X : \Omega\to\R \Big| \|X\|_{L^p} < \infty
        }
    \end{align}
\end{itemize}

\noindent Some basic inequalities and identities:
\begin{itemize}
    \item \textbf{1. Jensen's Inequality} - For a random variable $X$ and a convex function $\varphi:\R\to\R$, we have:
    \begin{align}
        \varphi(\E X) \le \E\varphi(X)
    \end{align}

    \item \textbf{2. Monotonicity of $L^p$ norm} - For a random variable $X$:
    \begin{align}
        \|X\|_{L^p} \le \|X\|_{L^q}, \ 0 \le p\le q \le \infty
    \end{align}

    \item \textbf{3. Minkowski's Inequality} - For $1\le p \le \infty$ and two random variables $X, Y$ in $L^p(\Omega, \Sigma, \Pm)$ space:
    \begin{align}
        \|X+Y\|_{L^p} \le \|X\|_{L^p} + \|Y\|_{L^p}
    \end{align}

    \item \textbf{4. Holder's Inequality} - For $p, q\in [1, \infty]$ such that $1/p + 1/q=1$. Then, for random variables $X\in L^p(\Omega, \Sigma, \Pm)$ and $Y\in L^q(\Omega, \Sigma, \Pm)$, we have:
    \begin{align}
        |\E XY| \le \|X\|_{L^p}\cdot\|Y\|_{L^q} 
    \end{align}
    
    \item \textbf{5. Markov's Inequality} - For a non-negative random variable $X$ and $t>0$, we have:
    \begin{align}
        \Pm(X \ge t ) \le \frac{\E X}{t}
    \end{align}

    \item \textbf{6. Chebyshev's Inequality} - For a random variable $X$ with mean $\mu$ and variance $\sigma^2$. Then, for any $t>0$, we have:
    \begin{align}
        \Pm(|X-\mu| \ge t) \le \frac{\sigma^2}{t^2}
    \end{align}

    \item \textbf{7. Integral Identity} - Let $X$ be a non-negative random variable, we have:
    \begin{align}
        \E X = \int_0^\infty \Pm(X > t)dt
    \end{align}
\end{itemize}

\noindent\subsection*{Exercises}
\begin{exercise}{Generalized Integral Identity}{exercise_1.1.1}
    Let $X$ be a random variable (not necessarily non-negative). Prove the following identity:
    \begin{align}
        \E X = \int_0^\infty \Pm(X > t)dt - \int_{-\infty}^0 \Pm (X<t)dt
    \end{align}
\end{exercise}

\begin{solution*}[Exercise \ref{ex:exercise_1.1.1}]
    For $x\in\R$, using the basic integral indentity, we have:
    \begin{align*}
        |x| &= \int_0^\infty \1{t < |x|}dt
    \end{align*}
    
    \noindent We consider the following cases:
    \begin{itemize}
        \item When $x < 0 \implies x = -|x|$:
        \begin{align*}
            x &= -\int_0^\infty \1{t < |x|}dt 
                = -\int_0^\infty \1{t < -x}dt = -\int_0^\infty \1{-t>x}dt = -\int_{-\infty}^0 \1{t>x}dt
        \end{align*}

        \item When $x\ge0 \implies x = |x|$:
        \begin{align*}
            x &= \int_0^\infty \1{t < |x|}dt = \int_0^\infty \1{t<x}dt
        \end{align*}
    \end{itemize}

    \noindent Therefore, for $x\in\R$, we can write:
    \begin{align*}
        x = \int_0^\infty \1{t<x}dt - \int_{-\infty}^0 \1{t>x}dt
    \end{align*}

    \noindent Therefore, for a random variable $X$ not necessarily non-negative, we have:
    \begin{align*}
        \E X &= \E\biggSquare{
            \int_0^\infty \1{t<X}dt - \int_{-\infty}^0 \1{t>X}dt
        } \\
        &= \E \int_0^\infty \1{t<X}dt - \E \int_{-\infty}^0 \1{t>X}dt \\
        &= \int_0^\infty \E\1{t<X}dt - \int_{-\infty}^0 \E\1{t>X}dt \\
        &= \int_0^\infty \Pm(t<X)dt - \int_{-\infty}^0 \Pm(t>X)dt
    \end{align*}
\end{solution*}

\begin{exercise}{$p^{th}$-moments via tails}{exercise_1.1.2}
    Let $X$ be a random variable and $p\in(0,\infty)$. Show that:
    \begin{align}
        \E|X|^p = \int_0^\infty pt^{p-1}\Pm(|X| > t)dt
    \end{align}
\end{exercise}

\begin{solution*}[Exercise \ref{ex:exercise_1.1.2}]
    
\end{solution*}
