\subsection{Convergence of Random Variables}
\newcommand{\SLUTSKY}{\hyperref[thm:slutsky_theorem]{(\mathrm{{\bf SLUTSKY}})}}

In this section, we revise the modes of convergence in random variables. 
\subsubsection{Convergence in Distribution}
\begin{definition}[Convergence in Distribution]
    Given a sequence of real-valued random variables $X_1, X_2, \dots$ with CDFs $F_1, F_2, \dots$. We say that the sequence converges in distribution to a random variable $X$ with CDF $F$, denoted $X_n\xrightarrow{d}X$ if:
    \begin{equation}
        \lim_{n\to\infty}F_n(x) = F(x),
    \end{equation}

    \noindent for all $x\in\R$ at which $F$ is continuous. Convergence in distribution can also be referred to as \underline{weak convergence} in measure theory.
\end{definition}

\begin{theorem}{Slutsky's Theorem $\SLUTSKY$}{slutsky_theorem}
    Let $X_n$ and $Y_n$ be two sequences of random variables such that $X_n\xrightarrow{d}X$ and $Y_n\xrightarrow{d} c$ where $c<\infty$ is a constant. Then, we have:
    \begin{enumerate}
        \item $X_n + Y_n \xrightarrow{d} X + c$.
        \item $X_nY_n \xrightarrow{d} cX$.
        \item $X_n/Y_n \xrightarrow{d} X/c$ if $c \ne 0$. 
    \end{enumerate} 
\end{theorem} 

\begin{proof*}[Theorem \ref{thm:slutsky_theorem}]
\end{proof*} 

\subsubsection{Convergence in Probability}
\begin{definition}[Convergence in Probability]
    Given a sequence of real-valued random variables $X_1, X_2, \dots$. We say that the sequence converges in probability to a random variable $X$, denoted $X_n\xrightarrow{p}X$ if:
    \begin{equation}
        \lim_{n\to\infty} \Pm\bigRound{|X_n - X| \ge \epsilon} = 0, \quad \forall \epsilon > 0.
    \end{equation}

    \noindent We also refer to convergence in probability as \underline{convergence in measure} in measure theory.
\end{definition}

\begin{proposition}{$X_n\xrightarrow{p}X\implies X_n\xrightarrow{d}X$}{probability_conv_implies_dist_conv}
    Let $X$ and the sequence $X_1, X_2, \dots$ be real-valued random variables. If $X_n\xrightarrow{p}X$, then $X_n\xrightarrow{d}X$.
\end{proposition}

\begin{proof*}[Proposition \ref{prop:probability_conv_implies_dist_conv}]
    We first prove the following claim: Let $X, Y$ be random variables, $a\in\R$ and $\epsilon>0$, the inequality $\Pm(Y\le a) \le \Pm(X \le a + \epsilon) + \Pm(|Y-X|\ge\epsilon)$ holds. We have:
    \begin{align*}
        \Pm(Y\le a) &= \Pm(Y\le a, X \le a + \epsilon) + \Pm(Y \le a, X \ge a + \epsilon) \\
            &\le \Pm(X \le a + \epsilon) + \Pm(Y - X\le a - X, a - X \le -\epsilon) \\
            &\le \Pm(X \le a + \epsilon) + \Pm(Y - X \le -\epsilon) \\
            &\le \Pm(X \le a + \epsilon) + \Pm(Y - X \le -\epsilon) + \Pm(Y - X \ge \epsilon) \\
            &= \Pm(X \le a + \epsilon) + \Pm(|Y - X| \ge \epsilon).
    \end{align*}

    \noindent Using the above inequality, we have:
    \begin{align*}
        \Pm(X \le a - \epsilon) - \Pm(|X_n - X| \ge \epsilon) \le \Pm(X_n \le a) \le \Pm(X\le a + \epsilon) + \Pm(|X_n - X| \ge \epsilon).
    \end{align*}

    \noindent Taking limits as $n\to\infty$ from both sides, we have:
    \begin{align*}
        F_X(a - \epsilon) \le \lim_{n\to\infty} F_{X_n}(a) \le F_X(a + \epsilon).
    \end{align*}

    \noindent Taking $\epsilon \to 0^+$, we have $\lim_{n\to\infty} F_{X_n}(a) = F_X(a)$.
\end{proof*}

\begin{proposition}{$X_n\xrightarrow{d}c\iff X_n\xrightarrow{p}c$}{dist_conv_to_const_implies_probability_conv}
    Let $c\in\R$ be a constant and $X_1, X_2, \dots$ be a sequence of real-valued random variables. Then, $X_n\xrightarrow{d}c\iff X_n\xrightarrow{p}c$.
\end{proposition}

\begin{proof*}[Proposition \ref{prop:dist_conv_to_const_implies_probability_conv}, \cite{book:hossien2014}]
    Since $X_n\xrightarrow{d}c$, we immediately have the following:
    \begin{align*}
        &\lim_{n\to\infty} F_{X_n}(c - \epsilon) = 0, \\
        &\lim_{n\to\infty} F_{X_n}(c + \epsilon / 2) = 1. 
    \end{align*}

    \noindent Then, for any $\epsilon>0$, we have:
    \begin{align*}
        \lim_{n\to\infty} (|X_n - c|\ge \epsilon) &= \lim_{n\to\infty}\Pm\bigSquare{\Pm(X_n \le c - \epsilon) + \Pm(X_n \ge c + \epsilon)} \\
            &= \underbrace{\lim_{n\to\infty} F_{X_n}(c-\epsilon)}_{=0} + \lim_{n\to\infty}\Pm(X_n\ge c+ \epsilon) \\
            &\le \lim_{n\to\infty} \Pm(X_n \ge c + \epsilon/2) \\
            &= 1 - \underbrace{\lim_{n\to\infty} F_{X_n}(c + \epsilon/2)}_{=1} \\
            &= 0.
    \end{align*}

    \noindent From the above, we have $\lim_{n\to\infty}\Pm(|X_n-c|\ge\epsilon)=0$ and $X_n\xrightarrow{p}c$.
\end{proof*}

\subsubsection{\texorpdfstring{Convergence in $L^p$ norm}{}}
\begin{definition}[Convergence in $L^p$ norm]
    Given a sequence of random variables $X_1, X_2, \dots$ and a real number $p\in[1, \infty)$. We say that the sequence converges in $L^p norm$ to a random variable $X$, denoted as $X_n\xrightarrow{L^p} X$ if:
    \begin{equation}
        \lim_{n\to\infty} \E|X_n - X|^p = 0.
    \end{equation}
\end{definition}

\begin{proposition}{$X_n\xrightarrow{L^p}X\implies X_n\xrightarrow{p}X$}{conv_in_lp_implies_conv_in_probability}
    Let $p\ge 1$ and $X_1, X_2, \dots$ be a sequence of real-valued random variables. Let $X$ be a random variable, then, $X_n\xrightarrow{L^p}X\implies X_n\xrightarrow{p}X$. 
\end{proposition}

\begin{proof*}[Proposition \ref{prop:conv_in_lp_implies_conv_in_probability}]
    Let $\epsilon>0$, we have:
    \begin{align*}
        \Pm(|X_n - X| \ge \epsilon) &= \Pm(|X_n - X|^p \ge e^p) \quad (p\ge 1) \\
            &\le \frac{\E|X_n - X|^p}{\epsilon^p}. \quad (\text{Markov's Inequality})
    \end{align*}

    \noindent Taking the limits from both sides, we have $\lim_{n\to\infty}\Pm(|X_n-X|\ge \epsilon) = 0$ and $X_n\xrightarrow{p}X$.
\end{proof*}

\subsubsection{Almost-sure Convergence}
\begin{definition}[Convergence almost-surely]
    Let $X_1, X_2, \dots$ be a sequence of real-valued random variables that map from a sample space $\Omega$. Let $X$ also be a real-valued random variable. We say that $X_n$ converges almost surely to $X$, denoted as $X_n\xrightarrow{a.s}X$, if:
    \begin{align*}
        \Pm\bigRound{
            \limsup_{n\to\infty} E_n
        } = 0 \quad \text{where} \quad E_n = \bigCurl{
            \omega\in\Omega: |X_n(\omega) - X(\omega)| \ge \epsilon
        }.
    \end{align*} 
\end{definition}

\begin{remark}[Consequence of $\BCL$]
    \begin{equation}
        \sum_{n=1}^\infty \Pm(E_n) < \infty \implies X_n\xrightarrow{a.s} X.
    \end{equation}
\end{remark}

\begin{proposition}{$X_n\xrightarrow{a.s}X\implies X_n\xrightarrow{p}X$}{a.s_conv_implies_conv_in_probability}
    Let $X_1, X_2,\dots$ be a sequence of real-valued random variables and also let $X$ be a real valued random variables. If $X_n\xrightarrow{a.s}X$ then $X_n\xrightarrow{p}X$.    
\end{proposition}

\begin{proof*}[Proposition \ref{prop:a.s_conv_implies_conv_in_probability}]
    Let $f_n:\Omega\to\R_+$ be a sequence of nonnegative Borel-measurable functions such that $f_n(\omega)=|X_n(\omega) - X(\omega)|$. By Fatou's Lemma (reverse), we have:
    \begin{align*}
        \underbrace{\Pm\bigRound{\limsup_{n\to\infty} \{\omega\in\Omega:|X_n(\omega) - X(\omega)| \ge \epsilon\}}}_{=0} &= \int f_nd\Pm \\ 
        &\ge \limsup_{n\to\infty}\Pm(|X_n - X|\ge\epsilon) \\
        &\ge \lim_{n\to\infty}\Pm(|X_n - X|\ge\epsilon).
    \end{align*}

    \noindent Hence, we have $\lim_{n\to\infty}\Pm(|X_n-X|\ge\epsilon)=0$ and $X_n\xrightarrow{p}X$.
\end{proof*}

\newcommand{\CMT}{\hyperref[thm:continuous_mapping_theorem]{(\mathrm{{\bf CMT}})}}
\begin{theorem}{Continuous Mapping Theorem $\CMT$}{continuous_mapping_theorem}
    Let $f:\R\to\R$ be a \underline{continuous} function and $X_1, X_2, \dots$ be a sequence of real-valued random variables. Then, the following statements hold true:
    \begin{enumerate}
        \item $X_n\xrightarrow{d}X \implies   f(X_n)\xrightarrow{d}f(X)$.
        \item $X_n\xrightarrow{p}X \implies   f(X_n)\xrightarrow{p}f(X)$.
        \item $X_n\xrightarrow{a.s}X \implies f(X_n)\xrightarrow{a.s}f(X)$.
    \end{enumerate}
\end{theorem}

\begin{proof*}[Continuous Mapping Theorem $\CMT$]
    Since almost-sure convergence implies the other two modes of convergence, we only have to handle the almost-sure convergence case. Since $h$ is continuous, for any $\omega\in\Omega$ such that $X_n(\omega)\to X(\omega)$, we have $f(X_n(\omega))\to f(X(\omega))$. Therefore, we have:
    \begin{align*}
        \bigCurl{\omega\in\Omega : X_n(\omega)\to X(\omega)}\subseteq\bigCurl{\omega\in\Omega:f(X_n(\omega))\to f(X(\omega))}.
    \end{align*}

    \noindent Therefore, we have:
    \begin{align*}
        &\Pm\bigRound{\limsup_{n\to\infty}\bigCurl{\omega\in\Omega:|f(X_n(\omega)) - f(X(\omega))| \le \epsilon}} \\
        &\ge \Pm\bigRound{\limsup_{n\to\infty}\bigCurl{\omega\in\Omega:|X_n(\omega) - X(\omega)| \le \epsilon}} = 1,
    \end{align*}

    \noindent for all $\epsilon>0$. Therefore, we have $f(X_n)\xrightarrow{a.s}f(X)$.
\end{proof*}
