%% ---------------- %%
\newcommand{\FacT}{\hyperref[thm:factorisation_theorem]{(\mathrm{{\bf FacT}})}}

\subsection{Properties of Point Estimators}
In this section, we will discuss ``point estimators", the desirable properties of classes of point estimators and important results. Firstly, let us define what a point estimator is:
\begin{definition}[Point Estimator]
    Let $\bX=(X_1, \dots, X_n)$ be a random $i.i.d$ sample from a distribution $p_{\btheta}$ over $\mathcal{X}$ that relies on parameter(s) $\btheta$. Let $\theta\in\Theta$ be a population parameter (population mean, variance, etc). A point estimator for $\theta$ is a mapping:
    \begin{equation}
        \widehat{\Theta}: \mathcal{X}^n \to \Theta,
    \end{equation}

    \noindent where $\widehat{\Theta}(\bX)$ serves as a ``best guess" of the quantity $\theta$ based on the available data. Due to the randomness of the sample, the estimator $\widehat\Theta(\bX)$ is a \textbf{random variable}.
\end{definition} 

\noindent In this note, we discuss some desirable properties of point estimators. To summarize these properties before getting into the technicality, we will illustrate the concepts with an example. Suppose that you are given a truck of apples and you want to guess the average weight $\theta$ of these apples. Suppose you have a method $\widehat{\Theta}$ to estimate $\theta$, the properties of unbiasedness, consistency, efficiency and sufficiency can be demonstrated as follows:
\begin{enumerate}
    \item \textbf{Unbiasedness}: Suppose that your method is to guess $\theta$ as the mean of the first $10$ apples you see. You repeat this guessing process over and over again and realize that the average of all your guesses matches the true value of $\theta$. So, on average, you guess $\theta$ correctly.
    \item \textbf{Consistency}: You take the first $5$ apples, your method $\widehat{\Theta}$ gives you an estimate a bit far off from $\theta$. You take 45 more apples, your guess gets closer to the true value of $\theta$. Now, you take $450$ more apples, your guess gets even closer. So, your guess gets closer and closer to the true value as you increase sample size.
    \item \textbf{Efficiency}: Suppose your friend has an estimation method $\widetilde{\Theta}$. It took you $50$ apples to get an estimate that is close to $\theta$ while it took your friend's method $500$ apples to get similarly close. So, your method is more efficient. 
    \item \textbf{Sufficiency}: Suppose someone gives you a summary table stating that you have $5000$ apples in the truck and the total weight is $100$kg. Then, you do not even have to measure any apple. The summary table already gives you sufficient information.
\end{enumerate} 

\subsubsection{Sufficiency}
\begin{definition}[Sufficient Statistics]
    Let $\bX = (X_1, \dots, X_n) \sim p_{\btheta}$ be a random sample drawn $i.i.d$ from a distribution with parameters $\btheta$. Let $\bU=T(\bX)$ be a statistic, then it is called a \underline{sufficient statistic} if the conditional distribution $p_{\bX|\bU}$ does not depend on $\btheta$.
\end{definition}

\begin{remark}[Intuition for ``Sufficiency"]
    To give a bit of intuition into what we consider a ``sufficient" statistics. Given a random sample $\bX$ from a distribution whose parameters are denoted as $\btheta$. A statistics $\bU=T(\bX)$ is \textbf{sufficient} for the inference of $\btheta$ iff \textbf{once we know $\bU$, there is no further information about $\btheta$ that we can derive from the data $\bX$}. For example, given $\bX=(X_1, \dots, X_n) \sim_\mathrm{i.i.d}\mathcal{N}(\mu, \sigma^2)$ and our task is to estimate the expectation $\mu$. Consider the following intuitive example:
    \begin{enumerate}[label=(\roman*)]
        \item $\bU_1 = \frac{1}{n}\sum_{i=1}^n X_i$: $\bU_1$ is a sufficient statistics because we have used up all the information from the sample $\bX$. More specifically, we have taken the sample mean of all available instances available. 
        \item $\bU_2 = X_1$: While it is unbiased like $\bU_1$, this is \underline{not a sufficient statistics} because there are further information (the remaining instances $X_2, \dots, X_n$) that is yet to be utilized in the random sample $\bX$.
    \end{enumerate} 
\end{remark} 

\begin{example}[Intuition for ``Sufficiency" - A Gaussian Illustration]
    Suppose that we are conducting a survey about children's height. Suppose that the distribution of heights follows a Gaussian distribution $\mathcal{N}(\mu, \sigma^2)$ and we are given a random sample of size $2$: $X_1, X_2\sim \mathcal{N}(\mu, \sigma^2)$. Then, we use the following statistics to make inference about $\mu$:
    \begin{enumerate}[label=(\roman*)]
        \item $\bU_1=X_1$, and
        \item $\bU_2=\frac{X_1 + X_2}{2}$. 
    \end{enumerate} 

    \noindent Now, we analyse the sufficiency of both $\bU_1, \bU_2$ by calculating the conditional density functions $p_{\bX|\bU_1}$ and $p_{\bX|\bU_2}$ as follows: Let $\bx=\{x_1, x_2 \}$ be a particular observation of the random sample $\bX$ and $\bu_1, \bu_2$ be values of $\bU_1$, $\bU_2$. We have

    \begin{enumerate}[label=(\roman*)]
        \item \textbf{Calculation of $p_{\bX|\bU_1}(\bx|\bu_1)$}: Obviously, when $x_1\ne\bu_1$, we have $p_{\bX|\bU_1}(\bx|\bu_1) = 0$. On the other hand, if $x_1=\bu_1$, we have
        \begin{align*}
            p_{\bX|\bU_1}(\bx|\bu_1) &= \Pm(X_1=x_1, X_2=x_2|\bU_1=\bu_1) \\
                &= \Pm(X_1 = \bu_1, X_2 = x_2) \\
                &= \frac{1}{2\pi\sigma^2}\exp\biggSquare{
                    -\frac{(\bu_1 - \mu)^2 + (x_2 - \mu)^2}{2\sigma^2}
                },
        \end{align*}

        \noindent which still depends on $\mu$. Therefore, $\bU_1$ is not a sufficient statistics. 


        \item \textbf{Calculation of $p_{\bX|\bU_2}(\bx|\bu_2)$}: Again, when $x_1+x_2\ne \bu_2$, $p_{\bX|\bU_2}(\bx|\bu_2)=0$. Suppose that $x_1+x_2=\bu_2$, using Bayes rule, we have
        \begin{align*}
            p_{\bX|\bU_2}(\bx|\bu_2) &= \Pm\biggRound{X_1=x_1, X_2=x_2\Bigg|\frac{X_1 + X_2}{2} = \bu_2} \\
            &= \frac{\Pm(\bU_2=\bu_2|X_1=x_1, X_2=x_2)\Pm(X_1=x_1, X_2=x_2)}{\Pm(\bU_2 = \bu_2)} \\
            &= \frac{\Pm(X_1=x_1, X_2=x_2)}{\Pm(\bU_2=\bu_2)}.
        \end{align*} 

        \noindent Note that $\bU_2 \sim \mathcal{N}\bigRound{\mu, \frac{\sigma^2}{2}}$. Hence, we have:
        \begin{align*}
             p_{\bX|\bU_2}(\bx|\bu_2) &= \frac{
                \frac{1}{2\pi\sigma^2}\exp\biggSquare{-\frac{(x_1 - \mu)^2 + (x_2 - \mu)^2}{2\sigma^2}}
             }{
                \frac{1}{\sqrt{\pi\sigma^2}}\exp\biggSquare{-\frac{(\bu_2 - \mu)^2}{\sigma^2}}
             } \\
             &= \frac{1}{2\sigma\sqrt{\pi}}\exp\biggSquare{
                -\frac{-(x_1 - \mu)^2 + (x_2 - \mu)^2 - 2(\bu_2 - \mu)^2}{2\sigma^2}
             } \\
             &= \frac{1}{2\sigma\sqrt{\pi}} \exp\biggSquare{-\frac{x_1^2 + x_2^2 - 2\bu_2^2}{2\sigma^2}}.
        \end{align*} 

        \noindent Since the above conditional density is independent of $\mu$, it is a sufficient statistics for $\mu$.
    \end{enumerate} 
\end{example} 

\begin{example}[Bernoulli random variables]
    Let $\bX = (X_1, \dots, X_n) \sim \bern(\theta)$ be a random sample from the Bernoulli distribution. Let $\bU = \frac{1}{n}\sum_{i=1}^n X_i$, then $\bU$ is a sufficient statistic of $\theta$. To illustrate this, suppose that $\bx = (x_1, \dots, x_n)$ is an observation of the random sample $\bX$ and $\bu = \frac{1}{n}\sum_{i=1}^n x_i$. We have:
    \begin{align*}
        \Pm(\bX=\bx| \bU=\bu) &= \frac{\Pm(\bX=\bx, \bU=\bu)}{\Pm(\bU=\bu)} \\
            &= \frac{\Pm(X_1=x_1, \dots, X_n=x_n, \sum_{i=1}^n X_i = \sum_{i=1}^n x_i)}{\Pm(\sum_{i=1}^n X_i = \sum_{i=1}^n x_i)} \\
            &= \frac{\Pm(X_1=x_1, \dots, X_n=x_n)}{\Pm(\sum_{i=1}^n X_i = \sum_{i=1}^n x_i)} \\
            &= \frac{\theta^{\sum_{i=1}^n x_i}(1-\theta)^{n-\sum_{i=1}^n x_i}}{\Pm(\sum_{i=1}^n X_i = \sum_{i=1}^n x_i)}.
    \end{align*}

    \noindent Now, setting $k=\sum_{i=1}^n x_i$, The denominator is basically the probability that the Bernoulli variables sums up to $k$. Hence, we can calculate the denominator as follows:
    \begin{align*}
        \Pm\biggRound{\sum_{i=1}^n X_i = k} &= \begin{pmatrix}
            n \\ k
        \end{pmatrix} \theta^k(1-\theta)^{n - k}.
    \end{align*}

    \noindent Therefore, we have:
    \begin{align*}
        \Pm(\bX = \bx | \bU = \bu) &= \frac{\theta^k(1-\theta)^{n-k}}{
            \begin{pmatrix}
                n \\ k
            \end{pmatrix} \theta^k(1-\theta)^{n - k}
        } = \frac{1}{
            \begin{pmatrix}
                n \\ k
            \end{pmatrix}
        }.
    \end{align*}

    \noindent Therefore, the conditional distribution does not depend on $\theta$ and $\bU$ is a sufficient statistic.
\end{example}

\begin{definition}[Sufficiency Principle]
    If $\bU=T(\bX)$ is a sufficient statistic for $\btheta$, then any inference about $\btheta$ should only depend on the sample $\bX$ through $\bU$. \color{blue}In other words, if we estimate $\btheta$ using an estimator $\hat\btheta$, only $\bU$ shows up in the formula of $\hat\btheta$, not the sample $\bX$ itself. We will see why this is the case in the Factorisation Theorem $\FacT$, which states that we can factorise the density function into a function of $\bU, \btheta$ and a function of the observations $\bx$ and thus, the inference about $\btheta$ is independent of the observations $\bx$\color{black}.
\end{definition}

\begin{theorem}{(Fisher-Neyman) Factorisation Theorem $\FacT$}{factorisation_theorem}
    Let $\bX = (X_1, \dots, X_n)$ be a random sample with joint density function $p_{\btheta}$ over $\mathcal{X}^n$. The statistic $\bU=T(\bX)$ is sufficient for the parameters $\btheta$ if and only if we can find functions $h, g$ such that:
    \begin{align*}
        p_{\btheta}(\bx) &= g(T(\bx), \btheta)h(\bx),
    \end{align*}

    \noindent for all $\bx\in\R^n$ and $\btheta\in\Theta$.
\end{theorem}

\begin{proof*}[Factorisation Theorem $\FacT$]
    We have to conduct the proof in both directions.
    \begin{itemize}
        \item $T(\bX)$ is sufficient $\implies$ Factorisation exists:
        Let $\bU=T(\bX)$ be a sufficient statistics and $\bu = T(\bx)$ be the statistics evaluated on the observations $\bx$. Then, we have:
        \begin{align*}
            p_{\btheta}(\bx) &= \Pm(\bX=\bx; \btheta) \\ 
                &= \Pm(\bX=\bx | \bU=\bu; \btheta) \Pm(\bU=\bu;\btheta).
        \end{align*} 

        \noindent Since $\bU=T(\bX)$ is a sufficient statistics, $\Pm(\bX=\bx|\bU=\bu;\btheta)$ does not depend on $\btheta$. Hence, we denote $h(\bx) = \Pm(\bX=\bx|\bU=\bu;\btheta)$. Furthermore, $\Pm(\bU=\bu;\btheta)$ is a function of $\bu$ and $\btheta$. We denote this function as $g(\bu, \btheta)$ and conclude that the factorisation $p_{\btheta}(\bx)=h(\bx)g(T(\bx),\btheta)$ indeed exists. 

        \item Factorisation exists $\implies$ $T(\bX)$ is sufficient: Suppose that there exists $g, h$ such that we have the factorisation $p_{\btheta}(\bx)=g(T(\bx), \btheta)h(\bx)$. We then have:
        \begin{align*}
            \Pm(\bX = \bx|\bU = \bu;\btheta) &=  \frac{p_{\btheta}(\bx)}{\Pm(\bU=\bu;\btheta)} = \frac{g(\bu, \btheta)h(\bx)}{\Pm(\bU=\bu;\btheta)}.
        \end{align*} 

        \noindent We denote $A_\bu = \bigCurl{\tilde\bx \in \mathcal{X}^n : T(\tilde\bx) = \bu}$. We have:
        \begin{align*}
            \Pm(\bU = \bu;\btheta) &= \sum_{\tilde \bx \in A_\bu} \Pm(\bX = \tilde\bx) \\ 
                &= \sum_{\tilde\bx\in A_\bu} p(\tilde\bx;\btheta) = \sum_{\tilde\bx\in A_\bu} g(T(\tilde\bx), \btheta)h(\tilde\bx) \\ 
                &= g(\bu, \btheta) \sum_{\tilde\bx\in A_\bu} h(\tilde\bx).
        \end{align*} 

        \noindent From the above, we have:
        \begin{align*}
            \Pm(\bX=\bx|\bU=\bu;\btheta) &= \frac{h(\bx)}{\sum_{\tilde\bx\in A_\bu} h(\tilde\bx)},
        \end{align*} 
        \noindent and the above expression does not depend on $\btheta$. Hence, $T(\bX)$ is a sufficient statistics.
    \end{itemize}
\end{proof*}

\subsubsection{Bias and Efficiency} 

\subsubsection{Consistent Estimator}

